\documentclass[11pt, a5paper]{exam}

%==============================================================================
% Pacotes
%------------------------------------------------------------------------------
\usepackage{euler}
\usepackage{fontspec}
 \setmainfont
 [
  Path           = fonts/,
  Extension      = .ttf,
  UprightFont    = *-Regular,
  BoldFont       = *-Bold,
  ItalicFont     = *-Italic
 ]{Alegreya}
\usepackage{polyglossia}
  \setdefaultlanguage[variant = brazilian]{portuguese}
\usepackage
 [
	 top    = 2.54cm,
  bottom = 2.54cm,
		left   = 1.91cm,
		right  = 1.91cm
	]{geometry}
\usepackage{coffee4}
\usepackage{graphicx}
	\graphicspath{{./figs/}}
\usepackage{float}
\usepackage[labelfont = bf, font = small]{caption}
\usepackage{fancybox}
\usepackage{booktabs}
\usepackage{multicol}
\usepackage{multirow}
\usepackage{amsmath, amsthm, amsfonts, amssymb, amscd}
\usepackage{mathtools}
\usepackage{systeme}
\usepackage{esint}  
\usepackage{cancel}
\usepackage{esvect}
\usepackage{array}
 \setcounter{MaxMatrixCols}{20}
\usepackage{xcolor}
\usepackage{hyperref} 																											
 \hypersetup
 {
  colorlinks  = true,
  linkcolor   = blue,
  citecolor   = blue,
  filecolor   = blue,
  urlcolor    = blue,
  pdfproducer = {LaTeX},
  pdfcreator  = {XeLaTeX},
  pdfauthor   = {Ícaro Vidal Freire},
  pdfsubject  = {Resposta para questão de Kedna}, 
  pdfkeywords = {Dúvida, Vetorial, Integral Dupla, Matemática, Física,
		               Volume, coordenada polar, coordenadas cilíndricas}
 }
\usepackage{lipsum}
%==============================================================================

%==============================================================================
% Operadores Matemáticos
%------------------------------------------------------------------------------
\DeclareMathOperator{\sen}{sen}
\DeclareMathOperator{\tg}{tg}
\DeclareMathOperator{\cossec}{cossec}
\DeclareMathOperator{\cotg}{cotg}
\DeclareMathOperator{\arcsen}{arcsen}
\DeclareMathOperator{\arctg}{arctg}
\DeclareMathOperator{\arcsec}{arcsec}
\DeclareMathOperator{\Ln}{Ln}
\DeclareMathOperator{\Arg}{Arg}
\DeclareMathOperator{\cis}{cis}
\DeclareMathOperator{\vol}{vol}
%==============================================================================

%==============================================================================
% Novos Comandos 
%------------------------------------------------------------------------------
\newcommand{\dd}{\,\mathrm{d}}
\newcommand{\vazio}{\varnothing}
\newcommand{\Abs}[1]{\vert{#1}\vert}
\newcommand{\versor}[1]{\cdot\vec{\textbf{#1}}}
\newcommand{\Cis}[1]{\cos{#1} + i\,\sen{#1}}
\newcommand{\intc}{\varointctrclockwise}
\newtheorem{teorema}{Teorema}
\newtheorem{obs}{Observação}
%==============================================================================

%==============================================================================
% Classe Exam
%------------------------------------------------------------------------------
\pointpoints{\,ponto}{\,pontos} %---> muda o nome dos pontos: singular e plural,
%                                     respectivamente
\pointformat{}
\qformat{\ovalbox{\bfseries Questão \thequestion}\hfill} %---> o hfill `quebra` 
%                                                              a linha
\renewcommand{\solutiontitle}{\noindent{\bfseries \itshape Resposta:}\enspace}%
\SolutionEmphasis{\small}
\shadedsolutions %-----------------------------------> Mostra sombra na solução
\definecolor{SolutionColor}{rgb}{.95,.95,.95}
\printanswers %--------------------------------------------> Mostra as Soluções
\pagestyle{headandfoot}
\runningfootrule
\firstpagefooter{}{\thepage}{}
\runningfooter{}{\thepage}{}
%==============================================================================

%==============================================================================
% Título
%------------------------------------------------------------------------------
\title{\textbf{O Argumento de Draper}}
\author{Amargosa-BA}
\date{2021.1}
%==============================================================================

%%%%%%%%%%%%%%%%%%%%%%%%%%%%%%%%%%%%%%%%%%%%%%%%%%%%%%%%%%%%%%%%%%%%%%%%%%%%%%%
% INÍCIO DO DOCUMENTO
%------------------------------------------------------------------------------
\begin{document}
%
\maketitle
%
\cofeAm{1}{.7}{94}{7cm}{2.5cm} %---------------> configuração da mancha de café
%
\begin{abstract}
Parte do livro \textit{Ciência, Religião e Naturalismo: onde está o conflito?},
que servirá como principal referência para preparação da apresentação 
``Críticas ao Argumento de Draper'', inserida no \textit{I Seminário Contraponto
de Ciência e Fé}.
Separei o texto em secções e fiz algumas modificações na estrutura do mesmo, sem
alterar o conteúdo.
\end{abstract}
%

\section{Introdução}

Defendi até agora, contra Dawkins e Dennett, a tese de que a evolução e o teísmo
são compatíveis. 
No sentido que dou aos termos, significa que não há verdades tão evidentes a
ponto de tornar a conjunção da evolução com o teísmo incoerente no sentido 
lógico amplo. 
Poderia se afirmar, por outro lado, que, mesmo que as coisas sejam assim, a 
veracidade da evolução nos dá alguns motivos para rejeitar o teísmo: talvez a 
evolução seja um \textit{evidencia} contra o teísmo.
É exatamente essa a tese de Paul Draper
\footnote
{
  Paul Draper, 
  ``Evolution and the problem of evil'', 
  in: Louis Pojman; Michael Rea, orgs.,
  \textit{Philosophy of religion: an anthology},
  5.ed. (Belmont, California: Thomson Wadsworth, 2008).
}: 

\begin{quote}
  ``Vou demonstrar que certos fatos conhecidos dão mais apoio à hipótese do 
  naturalismo que à hipótese do teísmo porque temos consideravelmente mais 
  razões para esperar que eles sustentem a pressuposição de que o naturalismo 
  seja verdadeiro em lugar da pressuposição de que o teísmo seja verdadeiro.''
\end{quote}

Quais seriam esses ``fatos conhecidos''? 
Um deles, segundo Draper, é a evolução
\footnote
{
  Draper, op. cit., p. 208. No artigo, ele também cita, como se fossem fatos 
  conhecidos, o modo pelo qual a dor e o prazer se distribuem em nosso mundo e o 
  modo pelo qual a dor e o prazer se vinculam à sobrevivência e ao êxito 
  reprodutivo. 
  Tratarei aqui apenas da tese mencionada no corpo do texto; seção IV, abordarei
  da tese de Philip Kitcher de que o desperdício, a predação e a dor envolvidos 
  na evolução são evidencias contra o teísmo.
}: 

\begin{quote}
  ``Minha posição é que a evolução é uma evidência que favorece mais o naturalismo 
  que o teísmo.
  Há, em outras palavras, um bom argumento \textit{evidencial} favorável ao 
  naturalismo e contra o teísmo.'' 
\end{quote}

A hipótese é que a evolução é mais provável -- pelo menos duas vezes mais 
provável, diz Draper -- no naturalismo do que no teísmo.

\section{O argumento}

Seu argumento se desenrola como segue. 

Seja $ E $ a \textit{evolução} (ou seja, a proposição de que todas as formas de 
vida terrestres existentes na época atual surgiram por meio da evolução),  
$ T $ o \textit{teísmo} e $ N $ o \textit{naturalismo}. 
Draper argumenta que:

\begin{equation}\label{hip1}
  P(E | N) \text{ é muito maior que } P(E|T).
\end{equation}

Disso ele infere que, se nado o mais for igual quanto às evidências, o 
naturalismo é mais provável que o teísmo.
Considerando que o naturalismo é incompatível com o teísmo, segue-se que teísmo 
é improvável.

\section{Resposta ao argumento de Draper}

\subsection{Considerações Iniciais}

Suponhamos, no entanto, que, como pensa a maioria dos teístas que refletiram 
sobre o assunto, o teísmo seja não contingente: necessariamente verdadeiro ou 
necessariamente falso. 
Nesse caso, \eqref{hip1} não implica que o naturalismo seja mais provável que o 
teísmo; ao contrário, implica obviamente que o teísmo seja verdadeiro. 

Se o teísmo é não contingente e falso, segue-se que é necessariamente falso; a 
probabilidade de uma proposição contingente dada uma falsidade necessária é $ 1 $;
logo, $ P(E|T) $ é $ 1 $. 
Mas se, como alega Draper, $ P(E|N) $ é maior que $ P(E|T) $ segue-se que 
$ P(E|T) $ é menor que $ 1 $, logo, $ T $ não é necessariamente falso. 
Se $ T $ não é necessariamente falso, é necessariamente verdadeiro em razão de 
sua não contingencia. 

Então, se o teísmo é não contingente e \eqref{hip1} é verdadeira, o teísmo é não 
apenas verdadeiro como necessariamente verdadeiro. 

Draper, como é obvio, pressupõe que o teísmo seja contingente; logo, seu 
argumento não virá ao caso se o teísmo for não contingente. 
Mas deixemos essa limitação de lado e examinem os detalhes de seu interessante 
argumento.

\subsection{Análise do argumento}

O que diz o argumento? 
Seja $ S $ a proposição de que ``alguns seres vivos relativamente complexos não 
descendem de organismos unicelulares relativamente simples, mas foram criados 
independentemente por uma pessoa sobrenatural''; nesse caso, como assinala 
Draper, 

\begin{equation}\label{hip2}
 \begin{split}
  P(E|N) \text{ será muito maior que } P(E/T) \text{ se, e somente se, }\\
  P(\sim\! S | N) \times P(E|\sim\! S \& N) \text{ for muito maior que } \\
  P(\sim\! S | T) \times P(E | \sim\! S \& T).
 \end{split}
\end{equation}

Naturalmente, ele propõe, para demonstrar que $ P(E|N) $ é muito maior que 
$ P(E|T) $, que $ P(\sim\! S | N) \times P(E | \sim\! S \& N) $ é muito maior que 
$ P(\sim\! S | T) \times P(E | \sim\! S \& T) $.
Para demonstrar isso, ele pretende mostrar que: 

\begin{parts}
	 \part $ P(\sim\! S | N) $ é muito maior que $ P(\sim\! S |T) $ e que
  \part $ P(E | \sim\! S \& N) $ é \textit{pelo menos tão grande quanto} 
   $ P(E | \sim\! S \& T) $.
\end{parts}

Com respeito a (a), Draper argumenta que $ P(\sim\! S | N) $ é pelo menos
\textit{duas vezes maior} que $ P(\sim\!\!\! S | T) $. 
Se, como ele também afirma, {\small $ P(E |\! \sim\! S\& N) $ } é ao menos tão grande quanto 
$ P(E|\sim\!\! S \& T) $, a consequência é que $ P(E | N) $ é pelo menos duas vezes 
maior que $ P(E | T) $; isso é suficiente, segundo parece pensar Draper, para 
que se possa afirmar que $ P(E | N) $ é muito maior que $ P(E | T) $. 
Se seu argumento estiver correto, consequentemente $ P(E | N) $ é pelo menos 
duas vezes maior que $ P(E|T) $.

Suponhamos que isso seja verdade: o que de fato demonstra? 
Como Draper diz, \textit{se tudo o mais for igual quanto às evidências}, o 
teísmo é improvável. 
Mas é claro que essa igualdade de evidências é uma ficção. 
Acaso não há no contexto desse argumento uma série de outras probabilidades que 
favorecem o teísmo pelo menos no mesmo grau?

Por exemplo: seja $ V $ a proposição ``há vida na Terra''. 
Considerando as enormes dificuldades de perceber como a vida pode ter surgido 
unicamente em virtude das leis da Física, $ P(V | N ) $ é baixa. 
Mas $ P(V | T) $ não é baixa: ao contrário, é muito provável que o Deus do 
teísmo quisesse que houvesse vida, e várias formas de vida. 
Logo, $ P(V | T) $ é muito mais alta -- imagino que por várias ordens de 
magnitude -- que $ P(V | N) $. 

Do mesmo modo, seja $ I $ a proposição ``há seres inteligentes''; seja $ M $ a 
proposição ``há seres dotados de sentido moral''; seja $ C $ a proposição ``há 
criaturas que prestam culto a Deus''; $ P(I|T) $, $ P(M|T) $ e $ P(C|T) $ são, 
cada uma delas, muito mais altas que $ P(I|N) $, $ P(M|N) $ e $ P(C|N) $. 
O Deus do teísmo desejaria muito que houvesse criaturas semelhantes a ele por 
serem racionais e inteligentes; também desejaria, sem dúvida, que houvesse 
criaturas dotadas de sentido moral, capazes de diferenciar o certo do errado; e 
muito provavelmente desejaria que houvesse criaturas capazes de experimentar sua 
presença e motivadas a prestar culto por sua experiência da grandeza e bondade 
de Deus. 

\section{Conclusões}

Há muitos outros ``fatos conhecidos'' que conferem mais plausibilidade ao teísmo 
que ao naturalismo. 
Nesse caso, portanto, as evidencias mais favoráveis ao naturalismo que ao 
teísmo, citadas por Draper, são mais que contrabalançadas pelas evidências mais 
favoráveis ao teísmo que ao naturalismo.

%
\end{document}
%------------------------------------------------------------------------------
% FIM DO DOCUMENTO
%%%%%%%%%%%%%%%%%%%%%%%%%%%%%%%%%%%%%%%%%%%%%%%%%%%%%%%%%%%%%%%%%%%%%%%%%%%%%%%